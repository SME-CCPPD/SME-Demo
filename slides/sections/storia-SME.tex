%
% Author: Nicola Bernardini <nicb@sme-ccppd.org>
%
%

\setcounter{ms}{0}
\refstepcounter{ms}
\begin{frame}
    \frametitle<+->{Storia della Scuola di Musica Elettronica (\arabic{ms})}

	\begin{itemize}[<+- | alert@+->]

		\item La Scuola di Musica Elettronica del Conservatorio ``Cesare Pollini''
    ha un passato glorioso:

      \begin{itemize}[<+- | alert@+->]

        \item Fondata da Teresa Rampazzi nel 1972

        \item In ordine di tempo, \`e la terza creata in Italia \uncover<+->{(dopo quella
        di Firenze fondata da Pietro Grossi e quella di Torino fondata da
        Enore Zaffiri)}

        \item Collegata al Dipartimento d'Ingegneria dell'Informazione
        dell'Universit\`a di Padova sin dal 1974\uncover<+->{, \`e stata
        strumentale alla nascita del celebre \emph{Centro di Sonologia
        Computazionale}}

     \end{itemize}

	\end{itemize}

\end{frame}

\refstepcounter{ms}
\begin{frame}
    \frametitle<+->{Storia della Scuola di Musica Elettronica (\arabic{ms})}

	\begin{itemize}[<+- | alert@+->]

		\item Nel 1992 la cattedra di Musica Elettronica viene assegnata a Nicola Bernardini

    \item Nel 2001, a seguito della Riforma Moratti, vengono creati due
    indirizzi all'interno dello stesso corso:

      \begin{enumerate}[<+- | alert@+->]

        \item indirizzo compositivo

        \item indirizzo tecnico di sala di registrazione

      \end{enumerate}

    \item Il corso viene riformato per seguire il cosiddetto ``processo di Bologna''
            in un triennio di primo livello seguito da un biennio di secondo
            livello

    \item Nel 2009 viene creato il \emph{SaMPL} \emph{living--lab}, primo
    \emph{living--lab} al mondo a essere dedicato alla musica e ai musicisti

	\end{itemize}

\end{frame}
