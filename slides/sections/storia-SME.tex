%
% Author: Nicola Bernardini <nicb@sme-ccppd.org>
%
%

\mode<article>{
	\begin{itemize}[<+- | alert@+->]

		\item La Scuola di Musica Elettronica del Conservatorio ``Cesare Pollini''

      \begin{itemize}[<+- | alert@+->]

        \item \`e stata fondata da Teresa Rampazzi nel 1972

        \item ed \`e la terza creata in Italia, \uncover<+->{dopo quelle
        di Firenze e Torino.}

        \item Collegata al Dipartimento d'Ingegneria dell'Informazione
        dell'Universit\`a di Padova sin dal 1974\uncover<+->{, \`e stata
        strumentale alla nascita del \emph{Centro di Sonologia Computazionale}}

     \end{itemize}

	\end{itemize}
}

\setcounter{ms}{0}
\refstepcounter{ms}
\begin{frame}
    \frametitle<+->{Storia della Scuola di Musica Elettronica (\arabic{ms})}

    La Scuola di Musica Elettronica del Conservatorio ``Cesare Pollini''
    ha una storia gloriosa che risale al 1972.

\end{frame}

\mode<article>{
	\begin{itemize}[<+- | alert@+->]

		\item Nel 1992 la cattedra di Musica Elettronica \`e assegnata a Nicola Bernardini

    \item Nel 2001 vengono creati due indirizzi:

      \begin{enumerate}[<+- | alert@+->]

        \item compositivo

        \item tecnico di sala di registrazione

      \end{enumerate}

    \item Il corso viene riformato per far seguire al triennio di primo livello
    un biennio di secondo livello

    \item Nel 2009 viene creato \textbf{SaMPL}, primo \emph{living--lab} al mondo
    dedicato alla musica e ai musicisti

	\end{itemize}

  Un \emph{living-lab} \`e un laboratorio dove gli utenti (in questo caso: i
  musicisti) sono al centro della ricerca
}

\refstepcounter{ms}
\begin{frame}
    \frametitle<+->{Storia della Scuola di Musica Elettronica (\arabic{ms})}

	\begin{itemize}

    \item Nel 2001 vengono creati due indirizzi:

      \begin{enumerate}

        \item compositivo

        \item tecnico di sala di registrazione

      \end{enumerate}

    \item Nel 2009 viene creato \textbf{SaMPL}, primo \emph{living--lab} al mondo
    dedicato alla musica e ai musicisti

	\end{itemize}

\end{frame}
