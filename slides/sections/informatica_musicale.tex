%
% $Id: informatica_musicale.tex 8 2014-02-04 21:01:21Z nicb $
%
% Copyright (C) 2007 Nicola Bernardini nicb@sme-ccppd.org
% 
% This work is licensed under a Creative Commons License, and specifically the
% 
%   Creative Commons Attribution-ShareAlike 2.5 License
%   http://creativecommons.org/licenses/by-sa/2.5/legalcode
% 
% Check http://www.creativecommons.org/ for more information on
% Creative Commons Licenses and the Creative Commons Project.
%

\setcounter{ms}{0}
\refstepcounter{ms}
\begin{frame}
    \frametitle{Che cos'\`e l'informatica musicale (\arabic{ms})}

    \begin{itemize}[<+- | alert@+->]
        \item{Applicazioni delle tecnologie informatiche alla musica}
        \item{Tutto \`e cominciato con\dots}
            \begin{itemize}[<+- | alert@+->]
                \item{\dots la musica elettronica}
                \item{\emph{computer music}}
                \item{\emph{sound and music computing}}
            \end{itemize}
        \item{Informatica Musicale $==$ Sound and Music Computing}
    \end{itemize}
    
\end{frame}

\refstepcounter{ms}
\begin{frame}
    \frametitle{Che cos'\`e l'informatica musicale (\arabic{ms})}
    \pgfdeclareimage[width=\textwidth,interpolate=true]{smc}{\imagedir/soa}
    
    \begin{center}
        \pgfuseimage{smc}
    \end{center}
    
\end{frame}


\setcounter{ms}{0}
\refstepcounter{ms}
\begin{frame}
    \frametitle{Che cosa \emph{NON} \`e informatica musicale}

    \begin{itemize}[<+- | alert@+->]

        \item{Non \`e \emph{informatica musicale} la video--scrittura della musica}
            \uncover<2->{(tipografia informatizzata)}

        \item<3->{L'informatica musicale non si occupa di \emph{tastiere
            surrogate}, anche se queste utilizzano numerosi risultati della
            ricerca in questo campo}

    \end{itemize}
    
\end{frame}
