%
% Author: Nicola Bernardini <nicb@sme-ccppd.org>
%
%

\setcounter{ms}{0}
\refstepcounter{ms}
\begin{frame}
    \frametitle<+->{La Scuola di Musica Elettronica -- oggi (\arabic{ms})}

	\begin{itemize}[<+- | alert@+->]

		\item Oggi, la Scuola di Musica Elettronica del Conservatorio ``Cesare Pollini''
          si articola cos\`i con due percorsi di studio:

      \begin{itemize}[<+- | alert@+->]

        \item indirizzo compositivo

        \item indirizzo tecnico di sala di registrazione

     \end{itemize}

    \item i due indirizzi condividono la maggior parte degli insegnamenti,
          differenziandosi solo per alcuni di carattere
          compositivo

	\end{itemize}

\end{frame}

\refstepcounter{ms}
\begin{frame}
    \frametitle<+->{La Scuola di Musica Elettronica -- oggi (\arabic{ms})}

	\begin{itemize}[<+- | alert@+->]

    \item Gli insegnamenti si dividono in materie:

	  \begin{itemize}[<+- | alert@+->]

      \item base
      
      \item caratterizzanti

      \item integrative o affini

      \item ulteriori

      \item a scelta dello studente

	  \end{itemize}

	\end{itemize}

\end{frame}

\refstepcounter{ms}
\begin{frame}
  \frametitle<+->{Materie di base}

	\begin{itemize}

    \item teoria della musica

    \item \emph{ear training}

    \item storia della musica

    \item storia della musica elettroacustica

    \item acustica musicale

    \item elettroacustica

    \item \ldots

	\end{itemize}

\end{frame}

\refstepcounter{ms}
\begin{frame}
  \frametitle<+->{Materie caratterizzanti}

	\begin{itemize}

    \item informatica musicale

    \item elaborazione numerica dei segnali

    \item programmazione

    \item esecuzione e interpretazione della musica elettroacustica

    \item composizione ed esecuzione della musica elettroacustica

    \item forme, sistemi e linguaggi musicali/tecniche compositive

	\end{itemize}

\end{frame}

\refstepcounter{ms}
\begin{frame}
  \frametitle<+->{Materie integrative o affini}

	\begin{itemize}

    \item composizione per la comunicazione visuale

    \item ambienti esecutivi multimodali e interattivi

    \item composizione audiovisiva integrata

    \item videoscrittura musicale ed editoria musicale informatizzata

	\end{itemize}

\end{frame}

\refstepcounter{ms}
\begin{frame}
  \frametitle<+->{Materie a scelta dello studente}

	\begin{itemize}

    \item piano formativo addizionale a carattere pratico (tirocinii \emph{SaMPL},
    tirocinii presso studi esterni convenzionati, \ldots)

    \item altre materie presenti nel piano d'offerta formativa del Conservatorio

    \item \ldots

	\end{itemize}

\end{frame}
