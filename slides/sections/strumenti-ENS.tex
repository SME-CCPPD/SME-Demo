%
% Author: Nicola Bernardini <nicb@sme-ccppd.org>
%
% Copyright (c) 2004 Nicola Bernardini
% Copyright (c) 2004 Conservatorio "C.Pollini", Padova
%
% This work is licensed under the Creative Commons 
% Attribution-ShareAlike License. To 
% view a copy of this license, visit 
% http://creativecommons.org/licenses/by-sa/2.0/ 
% or send a letter to Creative Commons, 
% 559 Nathan Abbott Way, Stanford, California 94305, USA.
%
% Some rights reserved.
% CVSId : $Id: strumenti-ENS.tex 8 2014-02-04 21:01:21Z nicb $
%
\setcounter{ms}{1}
\refstepcounter{ms}
\begin{frame}
    \frametitle<+->{Strumentario del corso (\arabic{ms})}

    \begin{center}
    La Scuola di Musica Elettronica promuove l'uso del\\[\baselineskip]

    {\Large {\bfseries Software Libero}}
    \vspace{\baselineskip}

    in tutte le attivit\`a musicali, creative e di ricerca.
    \end{center}

\end{frame}

\begin{frame}
    \frametitle{Strumentario del corso (\arabic{ms})}

	\begin{itemize}

    \item editing (\emph{audacity}, \emph{ardour},
                  \emph{Pro-Tools}, \emph{Logic}, \emph{Sadie}, \ldots)

    \item didattica (\emph{octave}, \emph{gnuplot}, \emph{geogebra}, \ldots)

    \item \emph{live--electronics} (\emph{PureData}, \emph{SuperCollider},
            \emph{Faust}, \emph{Max/MSP}, \ldots)


    \item \emph{sintesi} (\emph{csound}, \ldots)

	\end{itemize}

\end{frame}
